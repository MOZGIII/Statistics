










\subsection*{\hspace*{\parindent}Задание 4} 
По двум данным в задании 1 выборкам проверить гипотезу о некоррелированности 
соответствующих случайных величин с уровнем значимости $0{,}1$.

\par
{\em Решение.} 
Пусть на основании данных корреляционной таблицы найден выборочный 
коэффициент корреляции $r^*_{xy}$, который оказался отличным от нуля. 
Так как выборка отобрана случайно, то возникает вопрос о том, будет ли 
отличен от нуля теоретический коэффициент корреляции $r_{\xi\zeta}$. 
Необходимо при заданном уровне значимости $\alpha$ проверить гипотезу $H_0$ 
о том, что  при конкурирующей гипотезе $H_1$: $r_{\xi\zeta} \neq 0$. 
Если $H_0$ отвергается, то выборочный коэффициент корреляции значимо 
отличается от нуля, а случайные величины $\xi$ и $\zeta$ коррелированны.
Если же $H_0$ принимается, то делаем вывод, что выборочный коэффициент 
корреляции незначимо отличается от нуля, а случайные величины $\xi$ и 
$\zeta$ некоррелированны.

В качестве критерия для проверки $H_0$ выбирается случайная величина 
$$ T = r^*_{xy}\dfrac{\sqrt{n - 2\vphantom{{r^*_{xy}}^2}}}{\sqrt{1 - {r^*_{xy}}^2}}.$$

При справедливости гипотезы $H_0$ величина $T$ имеет так называемое 
распределение Стьюдента с $n-2$ степенями свободы. Критическая область 
для рассматриваемой гипотезы будет двусторонней, $T_{\text{крит}1} = 
-T_{\text{крит}2}$. Критическое значение $T_{\text{крит}2}$ определяется 
по заданным уровню значимости $\alpha$ и по числу степеней свободы $n-2$. 
Если $|T_{\text{набл}}| \geqslant t_{\text{крит}2}$, то гипотеза $H_0$ 
отвергается с уровнем значимости $\alpha$; 
при $|T_{\text{набл}}| < t_{\text{крит}2}$ нет оснований отвергнуть $H_0$ 
с уровнем значимости $\alpha$. 

Пусть $H_0$~--- нулевая гипотеза, что $r_{xy} = 0$, то есть что случайные 
величины некоррелированны; и $H_1$~--- альтернативная гипотеза: $r_{xy} \neq 0$.
Вычислим критерий:  
$$ 
T_{\text{набл}} = 
r^*_{xy}\dfrac{\sqrt{n - 2\vphantom{{r^*_{xy}}^2}}}{\sqrt{1 - {r^*_{xy}}^2}}.
$$



Выборочный коэффициент корреляции вычислен в задании 3: $r^*_{xy} = -0{,}515753$, 
$n = 25$, откуда:
$$
T_{\text{набл}} = 
-0{,}515753\cdot\dfrac{\sqrt{23}}{\sqrt{1 - (-0{,}515753)^2}} \approx
-2{,}887075.
$$

Имея в виду, что уровень значимости $\alpha$ и число степеней свободы для распределения Стьюдента равно $25-2=23$, найдем $T_{\text{крит}2} = 1{,}71387$.

 $|T_{\text{набл}}| \geqslant t_{\text{крит}2}$, cледовательно, гипотеза $H_0$ отвергается c уровнем значимости $\alpha = 0{,}1$.
 
\par
{\em Вывод.}
После проверки гипотезы о значимости выборочного коэффициента корреляции 
для выборки случайной величины $(\xi; \zeta)$, было сделано заключение о том, 
что гипотезу $H_0$ : $r_{\xi\zeta} = 0$ необходимо отвергнуть c уровнем 
значимости $\alpha = 0{,}1$. Это означает, что случайные величины 
коррелированны, выборочный коэффициент корреляции значительно 
отличается от нуля.






